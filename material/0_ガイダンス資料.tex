\documentclass[dvipdfmx,a4paper,11pt]{article}
\usepackage[utf8]{inputenc}
%\usepackage[dvipdfmx]{hyperref} %リンクを有効にする
\usepackage{url} %同上
\usepackage{amsmath,amssymb} %もちろん
\usepackage{amsfonts,amsthm,mathtools} %もちろん
\usepackage{braket,physics} %あると便利なやつ
\usepackage{bm} %ラプラシアンで使った
\usepackage[top=30truemm,bottom=30truemm,left=25truemm,right=25truemm]{geometry} %余白設定
\usepackage{latexsym} %ごくたまに必要になる
\renewcommand{\kanjifamilydefault}{\gtdefault}
\usepackage{otf} %宗教上の理由でmin10が嫌いなので


\usepackage[all]{xy}
\usepackage{amsthm,amsmath,amssymb,comment}
\usepackage{amsmath}    % \UTF{00E6}\UTF{0095}°\UTF{00E5}\UTF{00AD}\UTF{00A6}\UTF{00E7}\UTF{0094}¨
\usepackage{amssymb}  
\usepackage{color}
\usepackage{amscd}
\usepackage{amsthm}  
\usepackage{wrapfig}
\usepackage{comment}	
\usepackage{graphicx}
\usepackage{setspace}
\usepackage{pxrubrica}
\usepackage{enumitem}
\usepackage{mathrsfs} 
\usepackage{caption}

\setstretch{1.2}


\newcommand{\R}{\mathbb{R}}
\newcommand{\Z}{\mathbb{Z}}
\newcommand{\Q}{\mathbb{Q}} 
\newcommand{\N}{\mathbb{N}}
\newcommand{\C}{\mathbb{C}} 
\newcommand{\Sin}{\text{Sin}^{-1}} 
\newcommand{\Cos}{\text{Cos}^{-1}} 
\newcommand{\Tan}{\text{Tan}^{-1}} 
\newcommand{\invsin}{\text{Sin}^{-1}} 
\newcommand{\invcos}{\text{Cos}^{-1}} 
\newcommand{\invtan}{\text{Tan}^{-1}} 
\newcommand{\Area}{\text{Area}}
\newcommand{\vol}{\text{Vol}}
\newcommand{\maru}[1]{\raise0.2ex\hbox{\textcircled{\tiny{#1}}}}
\newcommand{\sgn}{{\rm sgn}}
%\newcommand{\rank}{{\rm rank}}



   %当然のようにやる.
\allowdisplaybreaks[4]
   %もちろん.
%\title{第1回. 多変数の連続写像 (岩井雅崇, 2020/10/06)}
%\author{岩井雅崇}
%\date{2020/10/06}
%ここまで今回の記事関係ない
\usepackage{tcolorbox}
\tcbuselibrary{breakable, skins, theorems}

\theoremstyle{definition}
\newtheorem{thm}{定理}
\newtheorem{lem}[thm]{補題}
\newtheorem{prop}[thm]{命題}
\newtheorem{cor}[thm]{系}
\newtheorem{claim}[thm]{主張}
\newtheorem{dfn}[thm]{定義}
\newtheorem{rem}[thm]{注意}
\newtheorem{exa}[thm]{例}
\newtheorem{conj}[thm]{予想}
\newtheorem{prob}[thm]{問題}
\newtheorem{rema}[thm]{補足}

\DeclareMathOperator{\Ric}{Ric}
\DeclareMathOperator{\Vol}{Vol}
 \newcommand{\pdrv}[2]{\frac{\partial #1}{\partial #2}}
 \newcommand{\drv}[2]{\frac{d #1}{d#2}}
  \newcommand{\ppdrv}[3]{\frac{\partial #1}{\partial #2 \partial #3}}


%ここから本文.
\begin{document}
%\maketitle

\newpage
\begin{center}
{\Large 2022年度秋冬学期 大阪大学 理学部数学科 \\ 幾何学基礎2(位相空間論) 演義} \\
 火曜4限(15:10-16:40) 理学部E210
\end{center}
\begin{flushright}
 岩井雅崇(いわいまさたか) 2022/00/00 \\
\end{flushright}
{\large 基本的事項}
\begin{itemize}
  \setlength{\parskip}{0cm} % 段落間
  \setlength{\itemsep}{0cm} % 項目間
\item この授業は対面授業です. 火曜4限(15:10-16:40)に理学部E210にて演習の授業を行います.
\item 授業ホームページ(\url{https://masataka123.github.io/2022_winter_generaltopology/})にて授業の問題等をアップロードしていきます. 
下に授業ホームページのQRコードを貼っておきます. 
\begin{figure}[htbp]
\begin{center}
 \includegraphics[height=30mm, width=30mm]{genetopo.png}
 \caption*{授業のQRコード}
\end{center}
\end{figure}
\end{itemize}

\hspace{-18pt}{\large 成績に関して}

次の1と2を満たしているものに単位を与えます.
\begin{enumerate}
  \setlength{\parskip}{0cm} % 段落間
  \setlength{\itemsep}{0cm} % 項目間
\item 幾何学基礎2の講義の単位が可以上である.
\item 14回授業時までに演習の授業で少なくとも1回以上発表している.
\end{enumerate}
ただし2の条件を達成できないものには別途救済レポートを課して2を達成したものとすることがある.

成績は講義の点数に演習の出来具合を加算してつける(と思います).

\vspace{11pt}
\hspace{-18pt}{\large 解答の仕方について}
\begin{itemize}
  \setlength{\parskip}{0cm} % 段落間
  \setlength{\itemsep}{0cm} % 項目間
  \item 問題の解答を黒板に書いて発表してください. 正答だった場合その問題はそれ以降解答できなくなります. 不正解だった場合他の人に解答権が移ります. 
  \item 複数人が解答したい問題があるときは平和的な手段で解答者を決めます. 
  %\item 英語問題を答える際には英語を和訳してください. なお解答は日本語で行っても良い.
  \item 演習問題の難易度は一定ではない. 難しい問題を解いた場合は成績に加点を行う. 
 \end{itemize}


\vspace{11pt}
\hspace{-18pt}{\large その他}
\begin{itemize}
  \setlength{\parskip}{0cm} % 段落間
  \setlength{\itemsep}{0cm} % 項目間
    \item 授業内容をあまり把握していないので, 演習問題と授業内容が噛み合ってない可能性があります.
  \item 休講あるいは補講をすることがあるので, こまめにホームページとCLEは確認してください.
    \item オフィスアワーを月曜15:00-17:00に設けています. この時間に私の研究室に来ても構いません(ただし来る場合は前もって連絡してくれると助かります.)
    \item $\pi$-base \url{https://topology.jdabbs.com}も活用してください. 
 \end{itemize}
 
\newpage
\begin{center}
{\Large 集合と位相の問題の示し方}
\end{center}
集合と位相の基本的な問題が解けない場合は, 次のことが疎かになっていると思います.
\begin{enumerate}
  \setlength{\parskip}{0cm} % 段落間
  \setlength{\itemsep}{0cm} % 項目間
\item 用語の定義は何かわかっていない.
%\item 仮定が何かわかっていない
\item 何を示すかわかっていない.
\item 論理的な展開がわかっていない.
\end{enumerate}
よって次の順に解決していけば良いと思います.
\begin{itemize}
  \setlength{\parskip}{0cm} % 段落間
  \setlength{\itemsep}{0cm} % 項目間
\item 用語の定義を"理解して"覚える. 初めは問題ごとに教科書で定義を見て行っても良い.
\item 何を示せば良いかゴールを明確にする
\item 何から何が導かれるのかきちんと論理的に考える. 大体の基本的な問題は打つ手が一個しかないので, 集合と位相の基本的な問題は結構簡単に証明できます.\footnote{これは簡単な詰将棋に似ています.} ちょっと難しい問題になると授業で示した定理が必要になるので, 定理の理解・暗記も必要になってきます.
\end{itemize}



\begin{exa}
%集合間の写像$f : X \rightarrow Y$と$X$の部分集合$A,B$について
$d$を$\R$上のユークリッド距離とする. $(-1,1)$は$(\R,d)$上で開集合であることをしめせ

-頭の中での思考-
\begin{itemize}
  \setlength{\parskip}{0cm} % 段落間
  \setlength{\itemsep}{0cm} % 項目間
\item 示すことは「$(-1,1)$は$(\R,d)$上で開集合である」こと. しかしこれでは示すことが具体的にはわからないので開集合の定義を書き下す.
\item 距離空間$(X,d)$の部分集合$V$が開集合であることの定義は「$V = V^i$となる」こと. しかし$V^i$の定義がわからないのでこの定義を書き下す.
\item $V^i$とは$V$の内点の集合. $a \in X$が$V$の内点であるとは, ある$\epsilon >0$があって$a$の$\epsilon$近傍
$$
N(a,\epsilon) = \{ x \in X | d(x,a) <\epsilon\}
$$
が$V$に含まれること.
\item 上の二つから$(-1,1)$が開集合であることを示すには「任意の点$a \in (-1,1)$が$(-1,1)$の内点であること」を示せば良い. (ここでようやく"何を示せば良いか具体的にわかった").
%これをもっと具体的に書き下すと「任意の$a \in (-1,1)$についてある$\epsilon$$(-1,1)$の内点である」ことを示せば良い.
 \item 「任意の点$a \in (-1,1)$が$(-1,1)$の内点であること」を示すには「ある$\epsilon >0$があって
 $$N(a,\epsilon) = \{ x \in \R | d(x,a)=|x-a| <\epsilon\} \subset (-1,1)$$
 となること」を示せば良い. これは$\epsilon-\delta$論法でやったことを真似れば良い.\footnote{もしかしたらここが一番難しいもしれない. ここは経験則になってしまう.}
 \end{itemize}
 ただこれを解答に書いてはいけない. これは頭の中の思考であり他人に見せるには"汚い"からである. 解答は次のようになると思う.
  
 \newpage
 [解答例.] 距離空間$(X,d)$の部分集合$V$が開集合であることの定義は「$V$の任意の点が$V$の内点である」ことである.\footnote{この一文はあってもなくても良い. あった方が採点者が試験で採点するときに"こいつは理解してそうだな"と思われるかもしれない.}よって任意$a \in (-1,1)$について$a$が$(-1,1)$の内点であることを示す. $a \in (-1,1)$について$\epsilon = 1 - |a|$とおくと
 $|x-a| < \epsilon$ならば
 $$
|x|=|x-0| \leqq |x-a| + |a-0| <\epsilon +|a| = 1
 $$
 である.\footnote{この$\epsilon$は超能力あるいは経験則を使って無理やり取ってきた.} 
 よって$a$の$\epsilon$近傍$N(a,\epsilon)=\{ x \in \R | |x-a| <\epsilon\}$について$N(a,\epsilon) \subset (-1,1)$である. つまり任意の点$a \in (-1,1)$について$a$は$(-1,1)$の内点であるので, $(-1,1)$は開集合である.\footnote{ここの部分はくどいかもしれない. まあ書いておいた方が解答として何を示したか分かり易いもする.}
\end{exa}

\vspace{11pt}
最後に大体の集合と位相の問題に言えることは次です.
\begin{center}
\underline{集合と位相の問題で簡単に示せないものには反例がある.}
\end{center}




 \end{document}
