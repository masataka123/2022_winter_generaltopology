\documentclass[dvipdfmx,a4paper,11pt]{article}
\usepackage[utf8]{inputenc}
%\usepackage[dvipdfmx]{hyperref} %リンクを有効にする
\usepackage{url} %同上
\usepackage{amsmath,amssymb} %もちろん
\usepackage{amsfonts,amsthm,mathtools} %もちろん
\usepackage{braket,physics} %あると便利なやつ
\usepackage{bm} %ラプラシアンで使った
\usepackage[top=30truemm,bottom=30truemm,left=25truemm,right=25truemm]{geometry} %余白設定
\usepackage{latexsym} %ごくたまに必要になる
\renewcommand{\kanjifamilydefault}{\gtdefault}
\usepackage{otf} %宗教上の理由でmin10が嫌いなので


\usepackage[all]{xy}
\usepackage{amsthm,amsmath,amssymb,comment}
\usepackage{amsmath}    % \UTF{00E6}\UTF{0095}°\UTF{00E5}\UTF{00AD}\UTF{00A6}\UTF{00E7}\UTF{0094}¨
\usepackage{amssymb}  
\usepackage{color}
\usepackage{amscd}
\usepackage{amsthm}  
\usepackage{wrapfig}
\usepackage{comment}	
\usepackage{graphicx}
\usepackage{setspace}
\usepackage{pxrubrica}
\usepackage{enumitem}
\usepackage{mathrsfs} 

\setstretch{1.2}


\newcommand{\R}{\mathbb{R}}
\newcommand{\Z}{\mathbb{Z}}
\newcommand{\Q}{\mathbb{Q}} 
\newcommand{\N}{\mathbb{N}}
\newcommand{\C}{\mathbb{C}} 
\newcommand{\Sin}{\text{Sin}^{-1}} 
\newcommand{\Cos}{\text{Cos}^{-1}} 
\newcommand{\Tan}{\text{Tan}^{-1}} 
\newcommand{\invsin}{\text{Sin}^{-1}} 
\newcommand{\invcos}{\text{Cos}^{-1}} 
\newcommand{\invtan}{\text{Tan}^{-1}} 
\newcommand{\Area}{\text{Area}}
\newcommand{\vol}{\text{Vol}}
\newcommand{\maru}[1]{\raise0.2ex\hbox{\textcircled{\tiny{#1}}}}
\newcommand{\sgn}{{\rm sgn}}
%\newcommand{\rank}{{\rm rank}}



   %当然のようにやる.
\allowdisplaybreaks[4]
   %もちろん.
%\title{第1回. 多変数の連続写像 (岩井雅崇, 2020/10/06)}
%\author{岩井雅崇}
%\date{2020/10/06}
%ここまで今回の記事関係ない
\usepackage{tcolorbox}
\tcbuselibrary{breakable, skins, theorems}

\theoremstyle{definition}
\newtheorem{thm}{定理}
\newtheorem{lem}[thm]{補題}
\newtheorem{prop}[thm]{命題}
\newtheorem{cor}[thm]{系}
\newtheorem{claim}[thm]{主張}
\newtheorem{dfn}[thm]{定義}
\newtheorem{rem}[thm]{注意}
\newtheorem{exa}[thm]{例}
\newtheorem{conj}[thm]{予想}
\newtheorem{prob}[thm]{問題}
\newtheorem{rema}[thm]{補足}

\DeclareMathOperator{\Ric}{Ric}
\DeclareMathOperator{\Vol}{Vol}
 \newcommand{\pdrv}[2]{\frac{\partial #1}{\partial #2}}
 \newcommand{\drv}[2]{\frac{d #1}{d#2}}
  \newcommand{\ppdrv}[3]{\frac{\partial #1}{\partial #2 \partial #3}}


%ここから本文.
\begin{document}
%\maketitle

\begin{center}
{\Large 6. 商位相}
\end{center}

\begin{flushright}
 岩井雅崇 2022/11/08
\end{flushright}

この問題を解答するにあたり以下の用語を定義しておく.(これは次回の演習の内容でもある).
   \begin{tcolorbox}[
    colback = white,
    colframe = green!35!black,
    fonttitle = \bfseries,
    breakable = true]
    \begin{dfn}
位相空間$(X, \mathscr{O})$とする.
$X$が\underline{ハウスドルフ空間(または$T_2$空間)}であるとは, 任意の$a, b \in X$について, ある$U, V \in \mathscr{O}$があって$a \in U, b \in V, U \cap V = \varnothing $となること.
  \end{dfn}
 \end{tcolorbox}
 
また断りがなければ, $\R^{n}$にはユークリッド位相を入れたものを考える. また$\R^{n+1}$の部分集合$S^n$を
$S^n = \{ (x_1, \ldots, x_{n+1}) \in \R^{n+1} \, |\,\sum_{i=1}^{n+1} x_{i}^{2} =1\}$
と定め, 位相は$\R^{n+1}$の相対位相を入れる. 

\begin{enumerate}[ label=\textbf{問}6.\arabic*]
	
	

\item 実数の集合$\R$について, 同値関係$\sim_{1}$を
	$$
	x \sim_{1} y \Leftrightarrow x - y \in \Z
	$$
	を考える. $\pi : \R \rightarrow \R / \sim_{1}$を標準写像とし$\pi$により$\R / \sim_{1}$に商位相を入れる. 以下の問いに答えよ.
	\begin{enumerate}
	\item $f : \R \rightarrow S^1$を$f(t) = (\cos 2 \pi t, \sin 2 \pi t)$とする. このときある連続写像$\tilde{f}: \R / \sim_{1} \rightarrow S^1$で$f = \tilde{f} \circ \pi $となるものが唯一存在することを示せ. 
	\item $\tilde{f}$は全単射であることを示せ. 
	%\item $f$は商写像であること示せ. 
	\item $\tilde{f}^{-1}$は連続であることを示せ. よって$\R / \sim_{1}$と$S^1$は同相である. \footnote{もし別に同相を示す手段があるなら他の方法を用いて良い. 実は授業後半の事実を用いると$(c)$は簡単に示せる. (おそらく現時点だと少々厄介である.).}
	\end{enumerate}

	
	%このとき$\R / \sim_{1}$は$S^1$と同相であることを以下

\item 実数の集合$\R$について, 同値関係$\sim_{2}$を
	$$
	x \sim_{2} y \Leftrightarrow x - y \in \Q
	$$
	とし$\R / \sim_{2}$に商位相を入れる.  $\R / \sim_{2}$はハウスドルフ空間であるか判定せよ. 

\item $X = \{(x,y) \in \R^2| \text{$y=0$または$y=1$} \}$とする. 同値関係$\sim$を
	$$
	(x_1,y_1) \sim (x_2,y_2) \Leftrightarrow \text{「$x_1 \neq 0$ かつ $x_1=x_2$」または「$y_1=y_2$かつ$x_1=x_2$」}
	$$
	とし$X / \sim$に商位相を入れる.  $X/ \sim$はハウスドルフ空間であるか判定せよ. 


 
\item 次の問いに答えよ. 
	\begin{enumerate}
	\item 閉写像でも開写像でない連続写像の例をあげよ.
	%\item 閉写像であるが開写像でない連続写像の例をあげよ.
	\item 連続全単射が開写像であれば同相写像であることを示せ.
	\end{enumerate}
	
\item$^*$ $(X, \mathscr{O}_X )$を位相空間とし, $\sim$を$X$上の同値関係とする. $\mathscr{O}(\pi)$を標準写像$\pi : X \rightarrow X/\sim$による商位相とし, $(X/\sim, \mathscr{O}(\pi))$を商空間とする. 
次の主張に関して, 真である場合は証明し, 偽である場合は反例をあげよ.
	\begin{enumerate}
	\item 商写像$\pi : X \rightarrow X/\sim$は開写像である.
	\item 商写像$\pi : X \rightarrow X/\sim$は閉写像である.
	\end{enumerate}
	
\item $\R^{n+1} \setminus \{ 0\}$について, 同値関係$\sim$を
	$$
	x \sim y \Leftrightarrow \text{0でない実数$\alpha$が存在して$x = \alpha y$}
	$$
	と定義する. 商写像$\pi : \R^{n+1} \setminus \{ 0\} \rightarrow (\R^{n+1} \setminus \{ 0\})/\sim$によって位相を入れたものを実射影空間と呼び, $ \R\mathbb{P}^{n}:= (\R^{n+1} \setminus \{ 0\})/\sim$と書く.  以下$x= (x_{1}, z_{2}, \ldots, x_{n+1})$を$\R\mathbb{P}^{n}$の元とみなしたものを$(x_{1}: \cdots : x_{n+1})$と書き実同次座標と呼ぶ. 
次の問いに答えよ
	\begin{enumerate}
	\item $i=1,\ldots, n+1$について$U_{i} = \{(x_{1}: \cdots : x_{n+1}) | x_i \neq 0\}$とおく. $\R\mathbb{P}^{n} = \cup_{i=1}^{n+1}U_i$であることをしめせ.
	\item $i=1,\ldots, n+1$について写像$f_i : \R^{n} \rightarrow U_i$を$f_i(y_1, \ldots,y_{n})=(y_1: \cdots :y_{i-1}:1:y_{i+1}:\cdots : y_{n} )$とする. $f_i : \R^{n} \rightarrow U_i$は同相写像を定めることを示せ.
	\end{enumerate}
\item $ \R\mathbb{P}^{1}$は$S^1$と同相であることを示せ.
\item 次の問いに答えよ.	
	\begin{enumerate}
	\item 
	$$
\begin{array}{ccccc}
\sigma: &S^{n}& \rightarrow & \R\mathbb{P}^{n}& \\
&(x_{1}, \ldots, x_{n+1}) & \longmapsto & 
(x_{1}: \cdots : x_{n+1})&
\end{array}
$$
は全射連続写像であることを示せ.
	\item $\sigma$は商写像であることを示せ. 
	\item 任意の$q \in \R\mathbb{P}^{n}$について$\sigma^{-1}(q)$の個数を求めよ.
	\item $f : S^2 \rightarrow \R^3$を$f(x,y,z)=(yz,zx,xy)$とする.  このときある連続写像で$\tilde{f}: \R\mathbb{P}^{2} \rightarrow \R^3$で$f =\tilde{f} \circ  \sigma$となるものが唯一存在することを示せ. 

	\end{enumerate}
\item$^{*}$
$GL(2, \R) $を$2 \times 2$の正則行列とする. $\begin{pmatrix}
a & b\\
c& d
\end{pmatrix}
\in GL(2, \R) $を$(a,b,c,d) \in \R^4$と同一視することで, $GL(2, \R)$を$\R^4$の部分集合とみなし, $\R^4$の相対位相を入れる.

$GL(2, \R) $に同値関係$\sim$を
$$
	A \sim B \Leftrightarrow \text{$P \in GL(2, \R)$が存在して$A = P^{-1} B P$}
	$$
を考える. 次の問いに答えよ.
	\begin{enumerate}
	\item 任意の$\alpha \neq 0$なる実数について
$\begin{pmatrix} 1& \alpha\\0& 1\end{pmatrix} \sim \begin{pmatrix} 1 & 1\\0& 1\end{pmatrix}$であることを示せ.
	\item 商空間$GL(2, \R)/\sim$はハウスドルフ空間であるか判定せよ.
	\end{enumerate}
	
%\item 教科書の例入れてみる?
%\item 次を示せ
%\item 商社像のuniversality 
%\item 幾何学1での問題を持ってくる.
%\item $S^1$と同相の問題
%\item 複素射影空間$\C\mathbb{P}^1$と$S^2$の同相
%\item なんか適当に割った空間のハウスドルフ性
 \end{enumerate}





 \end{document}
